\documentclass{article}

\usepackage[colorlinks=true,allcolors=blue]{hyperref}
\usepackage{booktabs}
\usepackage{rotating}

\newcommand{\gal}{\emph{Galilee}}
\newcommand{\tcse}{TCSE}

\title{Galilee:\\A Web-Based Scripture Engagement Application}
\author{Tom Nurkkala \and Ken Kiers}

\begin{document}
\maketitle

Throughout the 2015--16 academic year,
Phil Collins has worked with
Tom and his Information Systems students
to design and prototype
a web-based application
to support and encourage scripture engagement.
This document proposes initial work
on a production-ready implementation of
the key features of such a system,
to be created in partnership
with the Taylor Center for Scripture Engagement (\tcse).

The working name for the project
is \gal.\footnote{The source for the project name
  follows a circuitous path of brainstorming
  from ``Scripture Engagement Application''
  to ``SEA''
  to ``some relevant \emph{sea}''
  to ``\gal.''
  This is the name of the \emph{project},
  not necessarily the name of the \emph{application}.}
In the following, we
identify a prioritized list of features (section~\ref{sec:features}),
clarify key requirements (section~\ref{sec:requirements}),
suggest a strategy for managing the project (section~\ref{sec:management}),
and estimate the time and cost to implement \gal (section~\ref{sec:cost-time}).
Appendix~\ref{sec:defs} defines terms used with specific intent throughout.

\section{Features}
\label{sec:features}

The \emph{initial} features of \gal,
in priority order, are as follows.
\begin{enumerate}
\item A \emph{reading plan}
  that provides a structured collection of scripture \emph{passages}
  for engagement in a regular cycle.
  The initial plan will follow the Revised Common Lectionary
  (RCL),\footnote{See, for example, \url{http://lectionary.library.vanderbilt.edu/}.}
  a three-year cycle of scripture \emph{readings}
  followed by numerous denominations worldwide.
\item Written instruction and resources
  regarding scripture engagement \emph{practices}
  (e.g., Lectio Divina, journaling, memorization).
  Source material for these resources to be
  provided by
  the \tcse.\footnote{See \url{https://www.biblegateway.com/resources/scripture-engagement/}}
\item Ability to connect passages or readings
  to extra-biblical \emph{resources} including initially
  web sites,
  images,
  audio recordings,
  and video recordings.
\item Administrative oversight of the system
  that provides properly authenticated users
  with the ability to:
  \begin{enumerate}
  \item Add or revise scripture engagement practices
  \item Associate a specific practice with readings or passages
    to which the practice is applicable,
    including written guidelines and suggestions
    about how to apply the practice to a specific reading or passage.
  \item Add or revise resources (i.e., images, recordings, etc.)
    and associate those resources with specific readings or passages,
    along with \tcse-provided commentary on the relevance of the resource
    to the reading or passage.
  \item Manage users authorized to perform administrative functions.
  \item Manage members (item~\ref{i:member})
    who have registered with the system.
  \end{enumerate}
\item Public access to all plans, readings, passages, practices, and resources
  without requiring users to register.
\item\label{i:member} Capabilities available only to \emph{members},
  who are users who choose to register with the system.
  In addition to all the functionality available to
  unregistered, public users, a member can:
  \begin{enumerate}
  \item Keep a record of those readings or passages
    with which he or she has engaged using a practice.
  \item Write a private, searchable journal of reflections
    regarding readings, passages, practices, and resources.
  \item Receive an optional e-mail reminder to visit the site
    after a period of inactivity.
  \item Provide feedback regarding their experience with a practice
    (e.g., 1--5 ``stars,'', written message).
    This feedback would be available to authorized \tcse{} staff.
  \end{enumerate}
\end{enumerate}

\section{Requirements}
\label{sec:requirements}

In addition to the features enumerated in section~\ref{sec:features},
following are additional requirements for \gal.
\begin{enumerate}
\item Provide a public-facing interface
  that employs attractive, functional, and modern
  technologies and design practices.
\item Employ responsive design that supports users accessing the system
  with devices having varying screen resolutions,
  ranging from smart phones and tablets to laptop and desktop computers.
\item Track basic user behavior on the site
  to provide data for further analysis (e.g., using Google Analytics).
\item Require and support secure access to all administrative functions
  on the site.
\item Store all system data in a robust and reliable
  relational database management system.
\item Consult and collaborate with Bible Gateway (BG)
  to facilitate future integration or interoperability
  with BG systems, servers, and services.
\item Use technology, tools, and development practices
  that will allow the system to scale
  to handle traffic and request rates
  similar to the capabilities of BG.
\end{enumerate}

\section{Management}
\label{sec:management}

Historically,
projects like \gal{} have been managed using one of two strategies.
\begin{enumerate}
\item A \emph{fixed-price} contract
  requires detailed, fixed, \emph{a priori} negotiation of
  features, schedule, and cost.
  In practice, it is difficult to specify the feature details in advance
  and to predict the time and schedule prior to beginning work.
  Typically, a fixed-price contract disadvantages the development team
  when they discover that incompletely specified features
  take longer to implement than expected.
\item A \emph{time-and-materials} contract
  also includes \emph{a priori} negotiation of features
  but cost and schedule are allowed to float.
  Such a contract often disadvantages customers,
  who incur a vague financial liability
  that could change dramatically should the project run long.
\end{enumerate}
As an alternative to these traditional project management strategies,
we propose to develop \gal{}
using an agile, iterative process
that circumvents traditional problems
by replacing a negotiated contract
with a collaborative process that
encourages frequent communication between the \gal{} team and the \tcse.
Under such a process,
the \tcse{} retains flexibility in prioritizing features
and controls its financial liability
(difficult under a time-and-materials contract).
Similarly, the \gal{} team gets the feedback it needs to be successful
and is compensated fairly for the work performed
(relatively rare under a fixed-price contract).
In particular, the \tcse{}:
\begin{enumerate}
\item Prioritizes features to be implemented next in \gal,
  altering priorities as necessary throughout the project
\item Reviews and approves the progress of the project
  regularly (every two weeks or so)
\item Controls cost by retaining authority to alter---or even terminate---the project
if requirements are not being met
\end{enumerate}
Concomitantly, the \gal{} development team:
\begin{enumerate}
\item Benefits from regular direction on how best
  to satisfy \tcse{} requirements throughout the project
\item Controls its internal work practices to optimize
  delivery of relevant \gal{} features
\item Receives fair compensation
  based on the actual time spent satisfying \tcse{} requirements.
\end{enumerate}

\section{Cost and Time}
\label{sec:cost-time}

Table~\ref{tab:time} on page~\pageref{tab:time} shows
an estimate of the number of hours to implement
the features and requirements described in
sections~\ref{sec:features} and~\ref{sec:requirements}.

At our proposed rate for \gal{} of \$40/hour,
the estimated 364~hours yield
a total cost of \$14,560.

Note that Table~\ref{tab:time}
represents a good-faith estimate
that could vary
based on changing requirements
or unforeseen technical issues.
As detailed in section~\ref{sec:management},
regular communication
between the development team and the \tcse{}
allows the \tcse{} to control features and cost
as it sees fit throughout the project.

\begin{sidewaystable}
  \centering
  \begin{tabular}{rrrrrrrrrr}
    \toprule
                 & \multicolumn{4}{c}{Application} & & & \multicolumn{2}{c}{Communication} & \\
    \cmidrule{2-5}
    \cmidrule{8-9}
                 & Model & View & Controller & Admin & Data Import & Tooling & TCSE & BG & Total \\
    Start-up     & 8     & 8    &            &       & 4           & 12      & 4    & 4  & 40    \\
    Reading Plan & 4     & 8    & 8          & 8     & 24          &         & 4    & 4  & 60    \\
    Practices    & 8     & 24   & 8          & 16    & 8           &         & 12   &    & 76    \\
    Resources    & 12    & 32   & 16         & 12    &             & 16      & 12   & 4  & 104   \\
    Members      & 8     & 24   & 24         & 16    &             &         & 8    & 4  & 84    \\
    \addlinespace
    Total        & 40    & 96   & 56         & 52    & 36          & 28      & 40   & 16 & 364   \\
    \bottomrule
  \end{tabular}
  \caption{Time estimate for \gal{} (hours)}
  \label{tab:time}
\end{sidewaystable}

\appendix

\section{Definitions}
\label{sec:defs}

Some terms in this document are used with specific technical intent
as follows.

\begin{description}
\item[Member] A user who elects to register with the system.
\item[Passage] Contiguous Bible verses
 that form one element of a reading.
\item[Practice] Shorthand for a scripture engagement practice
  such as Lectio Divina or journaling.
\item[Reading] One section of a reading plan,
  such as the readings for a week or a day.
  A reading consists of one or more passages.
\item[Reading Plan] A long-term plan of readings
  such as the Revised Common Lectionary or the One Year Bible.
  A plan consists of one or more readings.
\item[Resource] An extra-biblical asset such as an image or recording.
\end{description}

\end{document}

%%% Local Variables:
%%% mode: latex
%%% TeX-master: t
%%% End:

%  LocalWords:  TCSE RCL Lectio Divina
