\documentclass{article}

\usepackage[colorlinks=true,allcolors=blue]{hyperref}

\newcommand{\gal}{\emph{Galilee}}

\title{Galilee:\\A Web-Based Scripture Engagement Application}
\author{Tom Nurkkala \and Ken Kiers}

\begin{document}
\maketitle

Throughout the 2015--16 academic year,
Tom and his Information Systems students
worked with Phil Collins
to design and prototype
a web-based application
to support and encourage scripture engagement.
This document proposes initial work
on a production-ready implementation of
the key features of such a system,
to be created in partnership
with the Taylor Center for Scripture Engagement (TCSE).

The working name for the project
is \gal.\footnote{The source for the project name
  follows a circuitous path of brainstorming
  from ``Scripture Engagement Application''
  to ``SEA''
  to ``some relevant sea''
  to ``\gal.''
  This is the name of the \emph{project},
  not necessarily the name of the \emph{application}.}
In the following,
we identify a prioritized list of features (section~\ref{sec:features}),
clarify key requirements (section~\ref{sec:requirements}),
and estimate the time and cost to implement \gal (section~\ref{sec:cost-time}).
Appendix~\ref{sec:defs} defines terms used with specific intent throughout.

\section{Features}
\label{sec:features}

The initial features of \gal,
in priority order, are as follows.
\begin{enumerate}
\item A \emph{reading plan}
  that provides a structured collection of scripture \emph{passages}
  for engagement in a regular cycle.
  The initial plan will follow the Revised Common Lectionary
  (RCL),\footnote{See, for example, \url{http://lectionary.library.vanderbilt.edu/}.}
  a three-year cycle of scripture \emph{readings}
  followed by numerous denominations worldwide.
\item Written instruction and resources
  regarding scripture engagement \emph{practices}
  (e.g., Lectio Divina, prayer, journaling, memorization).
  Source material for these resources to be
  provided by
  the TCSE.\footnote{See \url{https://www.biblegateway.com/resources/scripture-engagement/}}
\item Ability to connect passages or readings
  to extra-biblical \emph{resources} including initially
  web sites,
  images,
  audio recordings,
  and video recordings.
\item Administrative oversight of the system
  that provides properly authenticated users
  with the ability to:
  \begin{enumerate}
  \item Add or revise scripture engagement practices
  \item Associate practices with specific readings or passages
    to which the practice is applicable,
    including written guidelines and suggestions
    about how to apply the practice to a specific reading or passage.
  \item Add or revise resources (i.e., images, recordings, etc.)
    and associate those resources with specific readings or passages.
  \end{enumerate}
\item Public access to all plans, readings, passages, practices, and resources
  without requiring users to register.
\item Capabilities available only to \emph{members},
  who are users who choose to register with the system.
  In addition to all the functionality available to
  unregistered, public users, a member can:
  \begin{enumerate}
  \item Keep a record of those readings or passages
    with which he or she has engaged using a practice.
  \item Write a private, searchable journal of reflections
    regarding readings, passages, practices, and resources.
  \item Receive an optional e-mail reminder to visit the site
    after a period of inactivity.
  \item Provide feedback on experience with practices,
    including the reading or passage with which the practice was employed.
  \end{enumerate}
\end{enumerate}

\section{Requirements}
\label{sec:requirements}

In addition to the features enumerated in section~\ref{sec:features},
following are additional requirements for \gal.
\begin{enumerate}
\item Provide a public-facing interface
  that employs attractive, functional, and modern
  technologies and design practices.
\item Employ responsive design that supports users accessing the system
  with devices having varying screen resolutions,
  ranging from smart phones and tablets to laptop and desktop computers.
\item Require and support secure access to all administrative functions
  on the site.
\item Store all system data in a robust and reliable
  relational database management system.
\item Consult and collaborate with Bible Gateway (BG)
  to facilitate future integration or interoperability
  with BG systems, servers, and services.
\item Use technology, tools, and development practices
  that will allow the system to scale
  to handle traffic and request rates
  similar to the capabilities of BG.
\end{enumerate}

\section{Cost and Time}
\label{sec:cost-time}



\appendix

\section{Definitions}
\label{sec:defs}

Some terms in this document are used with specific technical intent
as follows.

\begin{description}
\item[Member] A user who elects to register with the system.
\item[Passage] Contiguous Bible verses
 that form one element of a reading.
\item[Practice] Shorthand for a scripture engagement practice
  such as Lectio Divina or journaling.
\item[Reading] One section of a reading plan,
  such as the readings for a week or a day.
  A reading consists of one or more passages.
\item[Reading Plan] A long-term plan of readings
  such as the Revised Common Lectionary or the One Year Bible.
  A plan consists of one or more readings.
\item[Resource] An extra-biblical asset such as an image or recording.
\end{description}

\end{document}

%%% Local Variables:
%%% mode: latex
%%% TeX-master: t
%%% End:

%  LocalWords:  TCSE RCL Lectio Divina
